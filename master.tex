%!TEX program = xelatex
\documentclass[12pt, a4paper, UTF8, fontset=adobe, oneside]{ctexbook} % oneside 去掉所有空白页

\linespread{1.3} % 行距设置
\setcounter{secnumdepth}{3} % 层次为3以上的标题能够生成序号

%% 宏包
\usepackage{amsmath} % AMS数学宏包
\usepackage{fancyhdr} % 设置页眉页脚宏包
\usepackage{geometry} % 设置页边距宏包
\usepackage{xcolor} % 颜色宏包
\usepackage{hyperref} % 交叉引用宏包 colorlinks启用彩色模式 参考文献引用为紫红色
\usepackage[listings,breakable]{tcolorbox} % 彩色盒子宏包 代码宏包
\usepackage{enumitem} % 枚举设置宏包
\usepackage{tikz} % 画图宏包
\usepackage[europeanresistors]{circuitikz} % 电路绘制宏包

% 宏包设置
% 页眉页脚样式
\pagestyle{fancy} % 页面样式采用fancyhdr宏包中的fancy
\fancyhf{} % 去掉页眉
\cfoot{\thepage} % 页脚中间显示页码
\renewcommand{\headrulewidth}{0pt} % 去掉页眉的横线
% 页边距设置
\geometry{top = 2.54cm, bottom = 2.54cm, left = 3.18cm, right = 3.18cm}
% 章节样式设置
\CTEXsetup[name={第,部分},number={\arabic{part}}]{part}
\CTEXsetup[name={第,章},number={\arabic{chapter}}]{chapter}
% 文档设置
\renewcommand\contentsname{目录} % 中文 目录
\renewcommand\bibname{参考文献} % 中文 参考文献
% 清华紫
\definecolor{THU}{RGB}{111, 23, 135}
% 交叉引用宏包
\hypersetup{colorlinks=true,linkcolor=THU,citecolor=THU}
% tcolorbox样式设置
\newtcolorbox{redbox}[2][]{colback=yellow!10,colframe=red!75!black,coltitle=white,fonttitle=\bfseries,fontupper=\kaishu,title=#2,#1,center title,center upper,breakable} % 红色
\newtcolorbox{magbox}[2][]{colback=yellow!10,colframe=magenta!75!black,coltitle=white,fonttitle=\bfseries,fontupper=\kaishu,title=#2,#1,center title,center upper} % 紫红色
\newtcolorbox{THUbox}[2][]{colback=yellow!10,colframe=THU!75!black,coltitle=white,fonttitle=\bfseries,fontupper=\kaishu,title=#2,#1,center title,breakable} % 紫罗兰色
\newtcolorbox{THUCbox}[2][]{colback=yellow!10,colframe=THU!75!black,coltitle=white,fonttitle=\bfseries,fontupper=\kaishu,title=#2,#1,center title,center upper,breakable} % 紫罗兰色 居中
\newtcolorbox{purbox}[2][]{colback=yellow!10,colframe=purple!75!black,coltitle=white,fonttitle=\bfseries,fontupper=\kaishu,title=#2,#1,center title,center upper} % 紫色
\usetikzlibrary{calc,shapes.multipart,chains,arrows,positioning} % tikz library
\tikzset{circarrow/.style={*->,shorten <=-2pt}}

\begin{document}
\frontmatter
\begin{titlepage}
\begin{center}

\vspace*{5cm}
% Title
{\huge \bfseries 电力拖动自动控制系统理论}\\[0.4cm]

\vspace{12cm}

% {\large NCEPRI} \\[0.3cm]
{\large 江浩} \\[1cm]
{\large \today}

\end{center}
\end{titlepage}

\begin{titlepage}
\begin{center}

\end{center}
\end{titlepage}

{
\hypersetup{linkcolor=black} % 目录链接为黑色
\pagenumbering{Roman} % 页码编号为大写罗马数字
\tableofcontents % 目录
}

\mainmatter % 正文部分 重新编号
\pagenumbering{arabic} % 页码编号为阿拉伯数字

\part{交流调速系统}

\chapter{基于稳态模型的异步电动机调速系统}

\begin{figure}
\begin{circuitikz}[scale=1.2]
\draw 
(0,0) node[anchor=east] {}
to [short, o-] (4,0)
(4.7,3) to [L=$L_m$, i>^=$\dot{I_0}$] (4.7,0)
(0,3) node[anchor=west] {}
to [short, o-] (0.2,3)
to [R=$R_s$, i=$\dot{I_s}$] (2.5,3)
to [L=$L_{ls}$] (4,3)
to [L=$L_{lr}^{'}$, i<=$\dot{I_r^{'}}$] (8,3)
(8,0) to [vR, l_=$\frac{R_r^{'}}{s}$] (8,3)
(8,0) to [short, -] (4,0);
\begin{scope}[>=stealth]
\draw [->] (0.2,2.5) -- node[right]{$\dot{U_s}$} (0.2,0.5);
\end{scope}
\end{circuitikz}
\end{figure}

\bibliographystyle{thubib}
\bibliography{refs}
\end{document}

%%% Local Variables:
%%% mode: xelatex
%%% TeX-master: t
%%% End:
